% !TEX TS-program = Xelatex
% !TEX encoding = UTF-8 Unicode

\documentclass[UTF8]{ctexart}
\usepackage{amsmath}
\usepackage[bottom]{footmisc}
\usepackage{geometry}
\usepackage{graphicx}
\usepackage{figsize}
\usepackage[separate-uncertainty = true]{siunitx}
\usepackage{tabu}
\usepackage{wasysym}
\geometry{left=0.7in,right=0.7in,bottom=0.7in,top=0.7in}

\title{实验二十二:迈克耳孙干涉仪}
\author{朱寅杰 1600017721}
\date{2017年10月27日}

\begin{document}

\maketitle

\section{调节方法与实验现象}

\subsection{调节仪器至工作状态}
我们先把激光器和扩束透镜等组装成一个可用于干涉的点光源。先目测将激光器粗调至接近水平,并且与干涉仪上的光具中心近似登高。然后将一个光阑放置在激光的出射方向上,调节光阑高度使得激光恰好能经小孔出射。将光阑平移至不同位置,迭代调节光阑高度与激光器的俯仰角,最终使得不论光阑距离激光器远近,光束都能恰好通过光阑小孔。如果在光阑上打出了衍射图样证明激光打到了小孔边缘,还应作微小调节。调节完毕之后,激光器与光阑就能产生一条水平准直的光束。

接下来要利用这个光阑调节干涉仪上两面平面镜$M_1$与$M_2$的角度朝向,使得它们相互垂直,都与分束板成\ang{45}。先将$M_1$与$M_2$背后的螺丝拧到方便前后调节的中间位置,然后仿照测视力的做法,用纸片盖住一面镜子,观察激光经另一面镜子反射后打在光阑上的位置,调节这面镜子背后的螺丝使得光阑上中心最亮的光点与光阑孔重合。调完一面镜子,再去调节另一面镜子,使得两面镜子的空间取向都符合实验要求,达到近似垂直的状态。

接下来要加入扩束透镜使得准直光束变为一个点光源发出的光,主要调节的是透镜与光束的高低共轴。先凭目测调节透镜高度与位置,使激光束能打进透镜镜头里,然后用一张纸在透镜像方作屏,移动纸以观察扩束所得圆斑是否在仍在原先准直方向上。调节透镜高度使得扩束圆斑的中心近似保持在原先的水平准直方向上,此时激光器、光阑与透镜三者等高,组装出了一个便于在干涉仪上工作的点光源。摇上干涉仪的光屏,应当能看到弧形的非定域条纹。
\subsection{非定域干涉}
在看到弧形的非定域条纹以后,微调$M_2$镜的空间朝向,将弧形条纹的中心移到屏上,即可看到一环环圆条纹。如果条纹过于密集,说明两个虚光源纵向距离依然较大,此时需要拧动$V_1$调节两个虚光源之间的距离,即可观察到较粗的圆条纹。

实验中可以观察到,在拧动$V_1$吞条纹的过程中(对应两个虚光源之间距离减小),屏幕上的条纹会慢慢变疏变粗。

在调出较粗的圆条纹之后,微调$M_2$镜的空间取向,将圆形条纹的中心移向远处,即可看见条纹由圆弧形向双曲线形直至直线转变。
\subsection{等倾干涉}
%定域干涉等倾条纹的调节方法,等倾条纹的变化规律及解释;
在调节出较粗的非定域圆条纹之后,$M_1$与$M_2$经分束板所成的像$M_2'$的平行度较好,此时在扩束透镜像方加入一片毛玻璃将点光源改为扩展光源,摇下干涉仪屏幕即可凭肉眼看到圆形的干涉条纹。但此时由于$M_1$与$M_2'$依然不算完全平行,形成的条纹尚不是严格的定域的等倾条纹,因此从视野不同位置去观察条纹看到的图样会不同,如果摇头晃脑会见到条纹吞吐现象。此时需要调节$M_2$梁下的两颗细调镜面角度的螺丝使得$M_1$与$M_2$严格平行。调节时由于移动视线时上下吞吐与左右吞吐相互独立,分别与镜面取向的两个自由度相耦合,因此如果在上下方向上条纹有吞吐只需调节$M_2$下方的细调螺丝,在左右方向上有吞吐只需调节$M_2$前方水平的细调螺丝。调节至上下左右移动视线条纹都不出现吞吐即证明已经调出了较严格的等倾干涉条纹,$M_1$与$M_2'$之间形成了一块平行度较好的虚薄膜。此时拧动$V_1$前后调节$M_1$可以看到条纹先变疏再变密,证明薄膜越薄等倾干涉条纹越稀疏。
\subsection{等厚干涉}
%定域干涉等厚条纹的调节方法,等厚条纹的变化规律及解释
在调出等倾干涉条纹之后,微微改变$M_2$镜子的空间角度使得虚薄膜不再严格平行,此时应能观察到弧形的条纹。拧动$V_1$使弧形条纹向其凹侧移动(即曲率圆心远离屏幕),可以观察到弧形条纹逐渐变直,在条纹拉直后某一时刻消失,再在\ang{180}反方向出现直条纹。此时$M_1$已经移过了$M_2'$,(已经变为楔形的)薄膜开始向反方向生长。

细调$M_2$的空间角度会发现条纹的疏密与方向会发生变化,这是因为等厚干涉条纹宽度反比于薄膜两面的夹角,薄膜越接近平行条纹越宽。

调出激光的等厚条纹,然后拧动$V_1$使得条纹移动到刚刚由弯曲变直的区域;打开台灯,并用灯罩遮住一半的扩展光源的红光,此时视野中应该能看到白光与红色条纹。准备好细调手轮$V_2$,向之前形成直条纹的中央区域移动。在某一时刻会观察到屏幕上出现彩色直条纹,这就是白光的等厚干涉条纹。撤去激光即可观察到完整的白底彩色条纹。从尺上读出彩色条纹中心对应的$M_1$位置为\SI{32.01042}{\mm}(最后一位为估读)。
%\section{压电陶瓷的压电常量}


\section{空气折射率的测定}
在已调出圆环非定域干涉的干涉仪的一条干涉光路上安装一个小气室。由于空气的折射率$n$与气压$p$成线性关系,当气压$p\rightarrow 0$时$n\rightarrow 1$,因此通过改变气室内气压即可改变气室内光程,从而造成干涉条纹的吞吐移动。实验时先对气室打气至约\SI{1500}{hPa},然后慢慢降低气室中的气压,在条纹吞吐整数环时记录此时的气压数值。数据记录如下表。
%由于中间亮环恰好将暗环吞下的瞬间较为难以捕捉到,因此为方便起见,当屏幕上最中间一环经过吞吐变为与前一次同样的大小时(由于屏上有标尺较为容易判断)视为完成了一条条纹的吞吐。但由于随着光程差的变化中间条纹的(blablabla)

\noindent
\begin{tabu} to \linewidth {X[1.5]|X X X X X X X X X X}
\hline
吞条纹数$n$&0 &1 &3 &4 &5 &6 &12 &14 &15 &16\\
\hline
气压$p$/hPa&1484 &1455 &1403 &1364 &1339 &1306 &1130 &1070 &1036 &1007\\
\hline
\end{tabu}

作线性回归,得到斜率为$\Delta p/\Delta N=\SI{-29.88}{hPa}$,相关系数$r=0.99987$,故而统计给出斜率的不确定度约为$\sqrt{\frac{1/r^2-1}{M-2}}=5.7\permil$。实验时气室长度$D=\SI{4.00}{\cm}$,激光波长$\lambda=\SI{632.8}{\nm}$,大气压取$p_0=\SI{101325}{Pa}$,则有$p_0$下的折射率
\begin{equation}
  n=1+p_0\frac{\lambda}{2D}\lvert\frac{\Delta N}{\Delta p}\rvert=\num{1.000268(2)}
\end{equation}
结果中不确定度的估计只计及了从多次测量数据中评估出的随机误差,未计及所给各仪器参数的不确定度与其他系统误差。
\end{document} 