% !TEX TS-program = Xelatex
% !TEX encoding = UTF-8 Unicode

\documentclass[UTF8]{ctexart}
\usepackage{amsmath}
\usepackage[bottom]{footmisc}
\usepackage{geometry}
\usepackage{graphicx}
\usepackage{figsize}
\usepackage[separate-uncertainty = true]{siunitx}
\usepackage{tabu}
\usepackage{wasysym}
\geometry{left=0.7in,right=0.7in,bottom=0.7in,top=0.7in}

\title{实验八:测定杨氏模量}
\author{朱寅杰 1600017721}
\date{2017年11月3日}

\begin{document}

\maketitle

\section{伸长法测杨氏模量}

用砝码配重拉伸钢丝,用显微镜放大钢丝的细小伸长,并用CCD摄像机(连接监视器屏)辅助观测。实验时先从\SI{200}{g}开始,每次加\SI{200}{g}的砝码读数,直至\SI{1800}{g},然后加上一个\SI{100}{g}的砝码,随后每次减一个\SI{200}{g}的砝码读数,直至\SI{100}{g}止。各个砝码的质量经过电子天平重新称量,同屏幕上的目镜读数一道整理于下表中。表中序号1~19的数字代表测量各组数据的先后顺序,测1~10时是加砝码的过程,测10~19时是减砝码的过程。
\begin{center}
\noindent
\begin{tabu} to 0.87\linewidth {X[-1] X[c] X[-10,c]|X[-10,c]|X[-10,c] X[-1,c] X[-1]}
\hline
\#	&砝码质量/g	&目镜读数$x$/mm	&隔900g逐差/mm	&目镜读数$x$/mm	&砝码质量/g	&\#
\\
\hline
19	&100.01 	&2.83 	&0.50 	&3.33 	&1000.07 	&5
\\
1	&199.93 	&2.89 	&0.50 	&3.39 	&1100.08 	&14
\\
18	&299.94 	&2.93 	&0.51 	&3.44 	&1200.04 	&6
\\
2	&399.60 	&3.00 	&0.50 	&3.50 	&1300.05 	&13
\\
17	&499.61 	&3.05 	&0.51 	&3.56 	&1400.08 	&7
\\
3	&599.71 	&3.10 	&0.51 	&3.61 	&1500.09 	&12
\\
16	&699.72 	&3.15 	&0.51 	&3.66 	&1600.51 	&8
\\
4	&799.74 	&3.22 	&0.50 	&3.72 	&1700.52 	&11
\\
15	&899.75 	&3.27 	&0.51 	&3.78 	&1800.63 	&9
\\
5	&1000.07 	&3.33 	&0.51 	&3.84 	&1900.64 	&10
\\
\hline
\end{tabu}
\end{center}

在测完钢丝拉伸量之后,对钢丝的长度与横截面积进行测量(这样也可以避免因为千分尺使钢丝形变对测量造成影响)。钢丝的长度使用带刀口的木尺测得为$L=\SI{80.28}{\cm}$,而木尺的允差为$e_L=\SI{1.5}{\mm}$,折算成标准差得$L=\SI{80.28(9)}{\cm}$。钢丝的横截面近似为圆形,用千分尺测量其不同位置不同方向上的直径,用平均直径来估计钢丝的横截面积。测得数据为

\noindent
\begin{tabu}{X[-10]|XXXXXXXXXX}
  \hline
  钢丝直径/mm&0.318	&0.312	&0.318	&0.310	&0.314	&0.316	&0.318	&0.317	&0.319	&0.317\\
  \hline
\end{tabu}

其平均值为\SI{.3159}{\mm},样本标准差为\SI{.003}{\mm}。千分尺的允差为\SI{.004}{\mm},零点位于\SI{-.002}{\mm}处。对平均值进行修正,并将仪器允差合成入不确定度中,得到$d=\SI{.318(2)}{\mm}$。

对于钢丝伸长量随砝码质量的变化,先用逐差法粗略处理,得到加减$m=\SI{900}{g}$砝码,目镜读数上平均移动$\overline{\delta l}=\SI{.506}{\mm}$,这一平均值的统计不确定度大约有$\sigma_{\overline{\delta l}}=\SI{.0016}{\mm}$。目镜读数的允差取最小分度\SI{0.05}{\mm},故而估计最后$\delta l$的不确定度为$\sigma_{\delta l}=\SI{.03}{\mm}$,即$\delta l=\SI{.51(3)}{\mm}$。而逐差法从砝码标称质量取的等间隔$m=\SI{900}{g}$,从实际称量结果实际上并不完全精确,最大的误差约有约\SI{.5}{g},按极限误差折算成标准差,得$m=\SI{900.0(3)}{g}$。于是粗略估计出杨氏模量$E=4mgl/\pi d^2\delta l=\SI{1.76(11)e10}{Pa}$。

当然用逐差法来处理数据实在是手边没有好的计算工具时的下策,我们正儿八经用最小二乘法来求算一下目镜读数$x$与$m$的关系。对这十九组数据作最小二乘法得到
$x=\SI{2.76971}{\mm}+m\times\SI{5.60266e-4}{\mm/g}$,相关系数$r=\num{.99986}$,由此估计出拟合所得斜率相对不确定度为$\sqrt{(1/r^2-1)/(N-2)}=\num{4e-3}$,因此$\delta_l/m=\SI{5.60(2)e-4}{\mm/g}$。结合其他数据及其不确定度计算出$E=4mgl/\pi d^2\delta l=\SI{1.77(2)e10}{Pa}$。

\section{挠度法测杨氏模量}
实验时将一条钢梁架在左右两个支架的刀口上,在钢梁中央加砝码配重使其弯曲,通过读数显微镜测量钢梁中央的微小升降,计算出梁的挠度,从而推算出钢梁的杨氏模量。

梁的宽度用游标卡尺一次测量,为$a=\SI{15.01}{\mm}$,厚度用千分尺测量了六次,分别为

\noindent
\begin{tabu} to 0.99\linewidth {X[-10]|XXXXXX}
\hline
钢板厚度$h$/mm&1.561	&1.539	&1.538	&1.525	&1.541	&1.531\\
\hline
\end{tabu}

其平均值为\SI{1.5391}{\mm},标准差为\SI{.012}{\mm}。千分尺的零点在\SI{.009}{\mm}处,允差按\SI{.004}{\mm}计,修正数值并合成不确定度得$h=\SI{1.530(5)}{\mm}$。

实验时曾多次改变两个刀口的距离$l$,分别取\SI{23.34}{\cm}、\SI{18.36}{\cm}、\SI{13.63}{\cm}(用钢尺测出)。分别对配重质量与梁中央的位移量的关系进行测量,数据如下表。
\begin{center}
\begin{tabu} to 0.99\linewidth {X[-1,c]X[c]|X[-1,c]X[c]|X[-1,c]X[c]}
\hline
\multicolumn2{c|}{$l=\SI{23.34}{\cm}$}	&\multicolumn2{c|}{$l=\SI{18.36}{\cm}$}	&\multicolumn2{c}{$l=\SI{13.63}{\cm}$}
\\
配重质量$m$/\si{\gram}&显微镜读数$z$/\si{\mm}&配重质量$m$/\si{\gram}&显微镜读数$z$/\si{\mm}&配重质量$m$/\si{\gram}&显微镜读数$z$/\si{\mm}\\
\hline
0	&41.578	&100	&41.508	&200	&41.735
\\
100	&41.393	&300	&41.277	&400	&41.629
\\
300	&40.821	&500	&40.966	&600	&41.518
\\
500	&40.363	&700	&40.728	&800	&41.411
\\
700	&39.802	&900	&40.487	&1000	&41.284
\\
900	&39.422	&800	&40.923	&900	&41.371
\\
1100	&38.859	&600	&41.201	&700	&41.501
\\
1000	&38.782	&400	&41.442	&500	&41.571
\\
800	&39.429	&200	&41.866	&300	&41.662
\\
600	&39.699	&0	&41.867	&	&
\\
400	&40.501	&	&	&	&
\\
200	&40.958	&	&	&	&
\\
\hline
\end{tabu}
\end{center}
从表中可以看出,读数显微镜鼓轮的空程差还是还是一如既往地大,甚至到了没“回程”只是单纯启停,都会拧空轮的地步(特别是\SI{18.36}{\cm}那组中从\SI{400}{g}减到\SI{200}{g}的时候)。但还是硬着头皮把加砝码和减砝码段分别做最小二乘法,采用两段的斜率的平均值作为采用的$\Delta z/\Delta m$的值,并计算杨氏模量$E=\Delta mgl^3/4\Delta zah^3$。回归分析与后续计算的结果列于下表中,其中斜率按照从相关系数估计出的相对不确定度$\sqrt{(1/r^2-1)/(N-2)}$来保留有效数字,后面杨氏模量计算值的有效数字也按照前面中间结果的不确定度保留。
\begin{center}
\begin{tabu} to 0.99\linewidth {X[-1]|X[c]X[c]|X[c]X[c]|X[c]X[c]}
\hline
&\multicolumn2{c|}{$l=\SI{23.34}{\cm}$}	&\multicolumn2{c|}{$l=\SI{18.36}{\cm}$}	&\multicolumn2{c}{$l=\SI{13.63}{\cm}$}\\
	&相关系数$r$	&$\frac{\Delta z}{\Delta m}/(\si{\mm\per\gram})$&相关系数$r$	&$\frac{\Delta z}{\Delta m}/(\si{\mm\per\gram})$&相关系数$r$	&$\frac{\Delta z}{\Delta m}/(\si{\mm\per\gram})$\\
\hline
加砝码&	-0.99918 	&-0.00248 	&-0.99885 	&-0.00130 	&-0.99944 	&-0.000560\\
减砝码&	-0.99205 	&-0.00248 	&-0.97456 	&-0.0013 	&-0.99213 	&-0.00052\\
斜率采用值&	&-0.00248	&	&-0.00130	&	&-0.00054\\
$E/(\SI{e11}{Pa})$&\multicolumn2{c|}{2.34}	&\multicolumn2{c|}{2.18}	&\multicolumn2{c}{2.1}\\
\hline
\end{tabu}
\end{center}
从上表的数据看,将各系列分成加减砝码的两段以后线性还是非常好的,增砝码与减砝码的斜率在统计不确定度范围内也相差不大。除了\SI{18.36}{\cm}的减砝码那组由于\SI{200}{\gram}那个outlier,删掉以后减砝码的数据点不足以支撑一个足够精确可靠的回归计算,因此舍弃减砝码的数据,只取加砝码段的数据作为最终数值。

按照理论计算,对于同一根钢梁取不同长度时应当有$\Delta z/\Delta m\propto l^3$。对此次实验测到的三对$l$与$\Delta z/\Delta m$的数值取对数,然后作线性回归得到$\Delta z/\Delta m\propto l^{2.8}$。出现这样的偏差的主要原因应该还是空程差造成鼓轮读数的误差十分大,导致不能得到准确的结果。如果数据有5\%到10\%的误差的话,那立方幂规律里面相差到0.2也是情有可原之事。今后实验遇到空程差严重的仪器,应当寻找方法与技巧去减小空程差对测量结果的影响。
\end{document} 