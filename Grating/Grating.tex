% !TEX TS-program = Xelatex
% !TEX encoding = UTF-8 Unicode

\documentclass[9pt,UTF8]{ctexart}
\usepackage{amsmath}
\usepackage[bottom]{footmisc}
\usepackage{geometry}
\usepackage{graphicx}
\usepackage{figsize}
\usepackage[separate-uncertainty = true,per-mode=symbol]{siunitx}
\usepackage{tabu}
\usepackage{wasysym}
\geometry{left=0.5in,right=0.5in,bottom=0.7in,top=0.7in}

\title{实验三十一:光栅特性与测定光波波长}
\author{朱寅杰 1600017721}
\date{2018年5月8日}

\begin{document}

\maketitle
\setcounter{section}{31}

\subsection{用汞绿光标定光栅的空间频率}

使用\SI{546.07}{\nm}的汞绿光来标定8号光栅的空间频率。实验时测量0、$\pm1$、$\pm2$级的衍射线的角位置,数据记录如下:
\begin{center}
\noindent
\begin{tabu} to \linewidth {X[c,-1]|X[c,-10] X[c,-10]|X[c,-10] X[c,-10]|X[c,-10] X[c,-10]|X[c,-10] X[c,-10]|X[c,-10] X[c,-10]||X[c,-10] X[c,-10]|X[c,-10] X[c,-10]}

\hline
\#	&$\theta_0$	&$\theta_0'$	&$\theta_1$	&$\theta_1'$	&$\theta_{-1}$&$\theta_{-1}'$	&$\theta_2$	&$\theta_2'$	&$\theta_{-2}$	&$\theta_{-2}'$	&$\bar{\phi_1}$&$\bar{\phi_{-1}}$	&$\bar{\phi_2}$&$\bar{\phi_{-2}}$
\\
\hline
1	&\ang{349;14;}	&\ang{169;12;}	&\ang{330;15;}	&\ang{150;16;}	&\ang{8;34;}	&\ang{188;33;}	&\ang{308;17;}	&\ang{128;18;}	&\ang{30;10;}	&\ang{210;12;}	&\ang{18;57.5;}	&\ang{19;20.5;}	&\ang{40;55.5;}	&\ang{40;58;}	
\\
2	&\ang{39;20;}	&\ang{219;20;}	&\ang{20;21;}	&\ang{200;22;}	&\ang{58;40;}	&\ang{238;38;}	&\ang{358;23;}	&\ang{178;27;}	&\ang{80;20;}	&\ang{260;16;}	&\ang{18;58.5;}	&\ang{19;19;}	&\ang{40;55;}	&\ang{40;58;}
\\
3	&\ang{103;7;}	&\ang{283;3;}	&\ang{84;7;}	&\ang{264;5;}	&\ang{122;30;}	&\ang{302;22;}	&\ang{62;10;}	&\ang{242;11;}	&\ang{144;8;}	&\ang{324;0;}	&\ang{18;59;}	&\ang{19;21;}	&\ang{40;54.5;}	&\ang{40;59;}
\\
\hline
\end{tabu}
\end{center}
由于不知名的原因(可能仪器调节不太好),$\pm1$级条纹的衍射角之间有一些偏差,所以算光栅空间频率的时候我们就直接用两条的平均值来算了。只计算随机误差,$\pm1$级的衍射角为\SI{19.154(5)}{\degree},$\pm2$级的衍射角为\SI{40.94444(8)}{\degree}。游标读数的允差按照\ang{;1;}计算,合成进去分别是\SI{19.154(10)}{\degree}和\SI{40.944(9)}{\degree}。用$\pm1$级算出的空间频率等于$\sin\phi/k\lambda=\SI{6.009(3)e5}{\per\meter}$,光栅刻线间距为\SI{1.664(1)e-5}{\meter};用$\pm2$级算出的空间频率等于$\SI{6.000(2)e5} {\per\meter}$,光栅刻线间距为\SI{1.6666(5)e-5}{\meter}。


\subsection{测量汞黄双线的波长}

\begin{center}
\noindent
\begin{tabu} to \linewidth {X[c,-10] X[c,-10]|X[c,-10] X[c,-10]|X[c,-10] X[c,-10]|X[c,-10] X[c,-10]|X[c,-10] X[c,-10]||X[c,-10] X[c,-10]|X[c,-10] X[c,-10]}

\hline
$\theta_0$	&$\theta_0'$	&$\theta_1$	&$\theta_1'$	&$\theta_{-1}$&$\theta_{-1}'$	&$\theta_2$	&$\theta_2'$	&$\theta_{-2}$	&$\theta_{-2}'$	&$\bar{\phi_1}$&$\bar{\phi_{-1}}$	&$\bar{\phi_2}$&$\bar{\phi_{-2}}$
\\
\hline
\ang{99;59;}	&\ang{279;53;}	&\ang{79;41;}	&\ang{259;39;}	&\ang{120;14;}	&\ang{300;8;}	&\ang{79;38;}	&\ang{259;33;}	&\ang{120;20;}	&\ang{300;12;}	&\ang{20;16;}	&\ang{20;15;}	&\ang{20;20.5;}	&\ang{20;20;}	
\\
\hline
\end{tabu}
\end{center}
用前面得到的光栅空间频率(取$\pm2$级的那个值$\SI{6.000(2)e5}{\per\meter}$)计算黄双线的波长。较红的一条按左右平均是\ang{20;15.5;},较紫的一条是\ang{20;20.25;},计入允差\ang{;1;}转成弧度是\num{.35357(17)}和\num{.35496(17)}。按照$\lambda=d\sin\phi$算出来两条线的波长分别是\SI{577.1}{\nm}和\SI{579.2}{\nm}。计入光栅常数和角度各自的不确定度,合成计算得到两条波长的值和它们的不确定度分别为\SI{577.1(2)}{\nm}和\SI{579.2(2)}{\nm},与书上给出的标准值\SI{576.96}{\nm}与\SI{579.07}{\nm}吻合。

角色散率$\Delta\phi/\Delta\lambda=\num{654843}$,其中波长差直接按书上标准值算。不确定度主要是两个角度测量的不确定度,每个角度测量不确定度都是\num{1.68e-4},从而得到两角差的相对不确定度为17.19\%,从而角色散率的值与不确定度为\num{6.5(11)e5}。

\subsection{用钠黄双线估计光栅的色分辨本领}
刚好无法分辨双线时狭缝宽度,使用读数显微镜测了三次,分别为$25.323-23.806=\SI{1.517}{\mm}$、$21.662-20.109=\SI{1.553}{\mm}$、$31.578-30.049=\SI{1.529}{\mm}$,平均为\SI{1.533}{\mm}。用色分辨本领的决定式计算,得到$R=\SI{1.533}{\nm}\times\SI{6.000e5}{\per\meter}=\num{920.1}$。钠黄双线分别为\SI{589.0}{\nm}和\SI{589.6}{\nm},按定义计算刚好能区分时分辨本领是$R=982$,两种算法算出的结果基本相符。

\subsection{习题}
使用公式$d\sin\phi=k\lambda$应保证是夫琅禾费衍射,入射光应是平行光,并且垂直入射光栅面。实验中通过调节分光计各部件使这个条件得到满足。入射光平行可由自准直法确认,垂直入射光栅面时绿十字像和中央零级亮线重合。

让光栅垂直平分$b_1b_2$连线的目的是让光栅的俯仰只由$b_1b_2$控制,让光栅条纹的取向只由$b_3$控制,两者独立后面调起来更快捷。不平分的话后面往返迭代调节有的你调的了。

光谱线不等高说明光栅刻线不与转轴平行咯。这样的话游标上读出的角度和实际的衍射角就不在同一个平面里,会有一个二阶的畸变,读出来的数就不太对了。

光栅的光谱是衍射角正弦正比于波长,三棱镜的色散光谱取决于玻璃材料的具体色散关系;并且光栅的光谱是对称的,棱镜色散的光谱只有一边。

在一个光栅下,如果两条谱线靠得太近无法分辨,那是因为衍射出的谱线本身有一个展宽,两条谱线离得太近以后如果距离与谱线展宽相当就混为一起分不出来了,再加放大系统也是分不清的。

分辨本领以及角色散率各自的两个计算式一个是定义式,一个是取决于光栅等的参数的决定式。
\end{document} 