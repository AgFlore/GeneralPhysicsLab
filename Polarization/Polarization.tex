% !TEX TS-program = Xelatex
% !TEX encoding = UTF-8 Unicode

\documentclass[UTF8]{ctexart}
\usepackage{amsmath}
\usepackage[bottom]{footmisc}
\usepackage{geometry}
\usepackage{hyperref}
\usepackage{graphicx}
\usepackage{figsize}
\usepackage[separate-uncertainty = true,per-mode=symbol]{siunitx}
\usepackage{tabu}
\usepackage{wasysym}
\geometry{left=0.7in,right=0.7in,bottom=0.7in,top=0.7in}

\title{实验二十一:观察光的偏振}
\author{朱寅杰 1600017721}
\date{2017年12月29日}

\begin{document}
\maketitle
\paragraph{使用偏振光镜验证布儒斯特定律}
我们首先粗调这个偏振光镜(仪器图见书上,此处略去不画了),使得下面的反射镜与竖直方向约夹\ang{33}角,(氦氖)激光器与水平方向夹\ang{24}角,此时光线即近似以玻璃的布儒斯特角(\ang{57})入射,并反射大致经过平台$T$和玻璃片堆$A$的中心。调节上面的镜片$A$使其也与竖直方向夹\ang{33}角。把一块偏振片放在平台$T$上,检验$P$反射出来的光,发现确实是线偏振光。同时我们也确认了这块偏振片的偏振方向。

将$A$绕着竖直轴旋转,手上拿着纸片接收$A$反射出的激光,观察其强弱变化(所以纸片的距离要离$A$上反射点的距离基本恒定)。实验中观察到反射光的强度随着$A$的旋转而强弱周期变化,在在$A$的法向与$P$的法向的水平投影接近垂直的时候(平台$Q$上读数分别为\ang{3}与\ang{357})强度最弱。而如果观察经过$A$的透射光的话(由于$A$的正上方不远处是一储物柜的底板,因此只需观察该底板上的光斑强度即可),会发现在$A$的法向与$P$的法向的水平投影在平行或反平行的时候(平台上$Q$读数为\ang{90}与\ang{270})强度最弱。

将$A$转动至前面反射光最弱的位置,然后调节$A$的俯仰(与竖直夹角),(仍然用纸片接收)观察反射光的强弱变化。发现仍是当$A$与$P$的俯仰角度相同的时候反射光最弱,转动$A$,随着$A$与$P$俯仰角之差的增加,反射光的强度都逐渐变强。而在$A$与$P$的法向的水平投影相互垂直的位置,调节$A$的俯仰,观察$A$的透射光,会发现其强度(在光能顺利透射的时候)(到了$A$接近竖直的时候光还是很弱的,可能都打到$A$的边缘上了)一直很强,看不出明显的强度变化。

要解释以上实验现象,首先我们知道当光以特定的一个布儒斯特角(等于$\tan^{-1}n$,是介质折射率的函数)入射到介质表面时,反射光将成为线偏振光,(电矢量的)偏振方向垂直于入射面。而$A$这个玻璃堆片也可以看作是一个能产生部分偏振光的“起偏器”,反射光中带有方向是垂直于入射面的线偏振成分,透射光中带有着平行于入射面的线偏振成分。现在入射$A$的是一束线偏振光,其偏振方向垂直于$P$的法向。现在光以布儒斯特角入射$A$:当$A$与$P$的法向的水平分量平行或反平行时,入射$A$的光的偏振成分全部垂直于入射面,这导致透射光(本应由平行入射面的偏振成分贡献)变弱,而反射光(本应由垂直入射面的偏振成分贡献)变强。当$A$的$P$的法向的水平分量相互垂直时,入射$A$的光的偏振成分平行于入射面,这导致反射光(本应由垂直入射面的偏振成分贡献)变弱,而折射光(本应由平行入射面的偏振成分贡献)变强。

现在将$A$转到前面反射光最弱,也就是$A$与$P$的法向的水平投影相互垂直的位置,入射$A$的光的偏振成分平行于入射面。在原来的布儒斯特角处,反射光完全由垂直于入射面的偏振成分贡献,因此原来的反射光较弱。当入射角开始远离布儒斯特角时,反射光中也出现了不垂直于入射面的偏振成分的贡献,因此反射光逐渐变强了。而由于人眼对弱光强的敏感度高于较强的光强,因而虽然伴随着反射光强的增大,透射光强应当是在减小的,然而本来透射光就比较强,而且减小的量并不多,人眼的敏感度不足以看出其变化。

\paragraph{观察方解石的双折射现象}
将一块方解石晶体I(普通解理面)平放在光源上,从上方观察,可以看到有两个光点。旋转方解石,可以发现其中一个光点基本固定,而另一个光点绕着此光点随着晶体转动。使用偏振片观察之,能确认产生两个光点的都是线偏振光,且偏振方向基本垂直。

将方解石晶体II的人工打磨面平放在光源上,从上方观察,可以看到从对面的人工打磨面出射的光只有一个光点,没有双折射现象出现。

我们知道方解石是单轴晶体,具有一个主光轴方向,当光线入射方向与主光轴方向不平行时即会出现双折射现象,产生的两道光中,一道是普通的折射光,另一道则会受到晶体对光的各向异性作用的影响(因而会随着晶体转动)。两道光都是线偏振光,且偏振方向相互垂直。而方解石II的人工打磨面的法向恰好就是其光轴方向,因而不会出现双折射的现象。
\paragraph{观察线偏振光通过$\lambda/2$片时的行为}
直接透过一个偏振片$P$看钠黄光光源,当$P$旋转的时候,观察到的光强并无变化。而在$P$与观察者之间再放上一个偏振片$A$以后,观察到的光强会随着$P$的旋转周期变化,$P$旋转一周能看到光强出现两次极大、两次基本消光。我们知道,普通光通过$P$后变为线偏振光,再通过一个偏振片后,原来的线偏振光只有平行于第二块偏振片的透偏方向的分量方能通过,因此光强正比于二者夹角的余弦平方,转过一个周期以后自然是两次极大两次消光。

将$P$与$A$调到消光位置,在中间插入一块$\lambda/2$片$C$。将$C$旋转一周,光强也随之周期变化,能看到四次极大、四次基本消光。在某一个消光的位置开始,将$C$稍许转过一个角度,将$A$转过一周,能观察到两次光强极大、两次消光。

转动偏振片$P$使其角度盘上的读数为\ang{0},撤去$C$,使$P$与$A$的偏振方向达到正交(消光)。此时$A$的角度盘上读数为\ang{80}。再插入$\lambda/2$片$C$,转动之使之消光。此时$C$的角度盘上读数为\ang{16}。从此时开始,保持$C$不动,将$P$转过某一角度$\theta_P$(此时消光被破坏),反方向转动$A$使系统重回消光,记录下此时$A$的角度。从中我们可以计算出线偏振光经过$C$时被转过的角度$\delta$。数据记录如下表(单位:度):
\begin{center}
\begin{tabu} to \linewidth {X[c]|X[c]X[c]X[c]}
\hline
$\theta_P$	&$\theta_A$	&$\theta_A-\theta_A(0)$	&$\delta=\theta_P-(\theta_A-\theta_A(0))$\\
\hline
0	&80	&0	&0\\
15	&69	&-11	&26\\
30	&52	&-28	&58\\
45	&36	&-44	&89\\
60	&22	&-58	&118\\
75	&7	&-68	&143\\
90	&352(=-8)	&-88	&178\\
\hline
\end{tabu}
\end{center}
从表中可以看出,偏振光经过$\lambda/2$片$C$后转过的角度约为$\theta$的两倍。若要作理论解释的话,我们知道,$\lambda/2$片能使得线偏振光的振动方向转过一个角度。具体说来,开始时的状态是不管插不插$\lambda/2$片都能达到消光,证明此时$\lambda/2$片未使偏振方向发生偏转,此时偏振方向在$\lambda/2$片的光轴上。$C$不动,$P$转过一个$\theta$角度,则此时入射光的偏振方向即与$\lambda/2$片的光轴夹了$\theta$角,相应地出射光的偏振方向将与$\lambda/2$片的光轴夹$-\theta$角,从而偏转方向转过了$2\theta$角度。这也解释了,当入射光偏振方向不变,$\lambda/2$片转过一周时,出射光的偏振方向将转过两周,从而透过$A$能看到四次消光的现象。

在固定$\lambda/2$片而转动$P$与$A$重回消光的时候,消光的效果往往没有前面固定$P$与$A$而转动$\lambda/2$片那么好。这或许是$\lambda/2$片的质量不够好(或者说入射光的单色性不够好?唔我觉得还是$\lambda/2$片不准的可能性大一些),导致出射的光的偏振方向并不完全落在$-\theta$上,而有一个小的分布,成为了有微弱短轴成分的椭圆偏振光。鉴于人眼对弱的光强更敏感,因此看上去“消光不干净”的观感就很明显了。当然这是我胡诌的,不能保证说的是对的,因为没有实验验证过。
\paragraph{用$\lambda/4$片产生椭圆偏振光}
转动$P$,仍使其角度盘读数为\ang{0},撤下$\lambda/2$片$C$,并调节$A$使之与$P$正交(消光),此时$A$的角度盘读数(理所应当)为\ang{80}。插入$\lambda/4$片,旋转之使之消光,此时$\lambda/4$片的角度盘读数为\ang{352}。和上一个实验一样,将$P$转过一个角度$\theta$,转动$A$一周观察光强的变化,从而确定经$\lambda/4$片出射后光的偏振状态。
\begin{center}
\begin{tabu}to \linewidth {X[c,-1]|X[c,-10]|X[c,-10]}
\hline
$\theta$	&$A$转过一周看到的现象	&光的偏振状态\\
\hline
\ang{0}	&在$A$位于\ang{80}与\ang{259}时出现明显消光现象	&接近线偏振光\\
\ang{15}	&在$A$位于\ang{75}与\ang{254}时出现光强极小值,消光不干净	&椭圆偏振光\\
\ang{30}	&光强随$A$的转动周期变化,能辨认出两次极小,但不够鲜明无法读出准确位置	&椭圆偏振光\\
\ang{45}	&仍能看到光强随$A$转动周期小幅度变化,但更加难以确定极小光强的准确位置	&椭圆偏振光\\
\ang{60}	&光强随$A$的转动周期变化,能辨认出两次极小,但不够鲜明无法读出准确位置	&椭圆偏振光\\
\ang{75}	&在$A$位于\ang{355}与\ang{178}时出现光强极小值,消光不干净	&椭圆偏振光\\
\ang{90}	&在$A$位于\ang{350}与\ang{172}时出现明显消光现象	&接近线偏振光\\
\hline
\end{tabu}
\end{center}
一开始调到消光时,相当于$\lambda/4$片没有改变入射光的偏振状态,此时入射光的偏振应该在$\lambda/4$片的一根主轴上。将入射光的偏振方向转过一个角度$\theta$,那么$\lambda/4$片会将出射光变为以自身两根主轴为长短轴的椭圆偏振光。当$\theta=\ang{45}$时应当会输出正椭圆偏振光,此时振幅的长轴与短轴的比值最接近1;当$\theta=\ang{90}$时入射光投到了另一条主轴上,此时应当重新输出线偏振光。对应到所做实验上,预期应当能看到各个$\theta$产生的椭圆偏振光的光强极大和极小的偏振方向是相同的,都在$\lambda/4$片的两条主轴对应的四个角度数值上。但实际观察中,由于读数不准(人工判断极小位置的不确定度还是蛮大的,要是把夫琅禾费衍射实验里那台光强传感器搬过来应该能准确不少)或是别的什么不为人知的原因(待考待考),各个角度之间有一定的偏移。
\paragraph{分辨椭圆偏振光与部分偏振光}
椭圆偏振光的产生(用一个$\lambda/4$片)及其性质在上面的实验中大家已经看到了。但是我们知道,部分偏振光(即自然光中夹杂着一定的线偏振成分)在检偏器下也具有着与椭圆偏振光同样的表现,都是检偏器转一周会有两个极大两个极小。但是借助一块$\lambda/4$片,我们还是有办法区分出它们的。(顺带说一下部分偏振光的产生,我们只需要将自然光经一个玻璃片堆折射后,其透射光即带有部分线偏振成分,是一个容易实现的部分偏振光的来源。)

检验的第一步,让光线透过一块偏振片$P_1$。转动$P_1$,不论是椭圆偏振光还是部分线偏振光都会能看到出射光强的周期变化。将$P_1$调到出射光强极小的位置。此时经$P_1$出射的是一个线偏振光。在$P_1$与观察者之间再插入一个偏振片$P_2$,调整角度使二者正交,把$P_1$出来的线偏振光消干净。

然后在$P_2$与$P_1$之间插入一块$\lambda/4$片,转动之使从$P_2$出射的光强仍然最弱。此时入射$\lambda/4$片的光与出射$\lambda/4$片的光是偏振方向相同的线偏振光,并且偏振方向在$\lambda/4$片的一根主轴上。

接下来,撤去$P_1$,将$P_2$转过一周,观察光强的变化。如果最开始是椭圆偏振光,其偏振的短轴对准了原先$P_1$的出射方向,也就对准了$\lambda/4$片的一根主轴,出射时候应当是线偏振光,转动$P_2$观察应当能看到明显的消光现象。如果最开始是部分偏振光,其线偏振成分与原先$P_1$的出射方向正交。由于$P_1$的偏振方向与$\lambda/4$片的一根主轴对准,因此原来的线偏振成分应当会与$\lambda/4$片的另一根主轴对准,出射的是线偏振光;而原来的杂光成分会无阻碍地通过$\lambda/4$片和偏振片,因此最后观察到的仍然是部分偏振光的检偏现象,即只能观察到光强变化而看不到完全消光。

实验实际操作时,如果入射的是椭圆偏振光,检验最后一步也会出现光消不干净的情况。这可能是因为椭圆偏振光产生时所用的偏振片不够好导致混入了一些杂光成分吧。事实上我发现,使用标签为$A$的偏振片拿来对标签为$P_1$、$P_2$的偏振片作正交消光的时候会不太干净,而$P_1$和$P_2$相互作正交消光的时候能消得比较干净。大概偏振片的质量确实有好有坏吧。待考待考。
\end{document} 