% !TEX TS-program = Xelatex
% !TEX encoding = UTF-8 Unicode

\documentclass[UTF8]{ctexart}
\usepackage{amsmath}
\usepackage[bottom]{footmisc}
\usepackage{geometry}
\usepackage{graphicx}
\usepackage{figsize}
\usepackage[separate-uncertainty = true,per-mode=symbol]{siunitx}
\usepackage{tabu}
\usepackage{wasysym}
\geometry{left=0.7in,right=0.7in,bottom=0.7in,top=0.7in}

\title{实验十四:直流电桥测电阻}
\author{朱寅杰 1600017721}
\date{2017年11月10日}

\begin{document}

\maketitle

\section{自组电桥测量未知电阻与电桥灵敏度}
拧动电阻箱$R_0$某一档位的旋钮使其阻值变化$\Delta R_0$,观察检流计指针的偏转量,从而计算出灵敏度$S=\frac{\Delta n}{\Delta R_x/R_x}=\frac{\Delta n}{\Delta R_0/R_0}$。

电源电压$E=\SI{4.030}{\volt}$。电阻箱上标称对于\SI{10}{\ohm}以上各档有0.1\%的允差,对\SI{1}{\ohm}档有0.5\%的允差,对\SI{0.1}{\ohm}档有2\%的允差。检流计的内阻为\SI{47}{\ohm},盘面上一格为\SI{1.3e-6}{\ampere}。
\begin{center}
\noindent
\begin{tabu} to 0.99\linewidth {X[c,-10] X[c,-10]|X[c] X[c] X[c]|X[c] X[c]}
\hline
待测电阻	&$R_1/R_2$	&$R_0/\si{\ohm}$	&$\Delta R_0/\si{\ohm}$	&$\Delta n$	&$R^x/\si{\ohm}$	&灵敏度$S$
\\
\hline
$R^x_1=\SI{48.0}{\ohm}$&	\SI{500.0}{\ohm}/\SI{500.0}{\ohm}&	48.3&	0.1&	4.4&	48.3	&\num{2.1e3}
\\
\hline
$R^x_2=\SI{365.3}{\ohm}$&	\SI{50.0}{\ohm}/\SI{500.0}{\ohm}&	3661.5&	30&	4.9&	366	&\num{6.0e2}
\\
	&	\SI{500.0}{\ohm}/\SI{500.0}{\ohm}&	366.1&	1&	4.2&	366.1	&\num{1.5e3}
\\
	&	(交换$R_1$与$R_2$)&	366.2&	1&	4.2&	366.2	&\num{1.5e3}
\\
\hline
$R^x_3=\SI{3978}{\ohm}$&	\SI{500.0}{\ohm}/\SI{500.0}{\ohm}&	3988&	100&	7.7	&3988	&\num{2.1e3}
\\
\hline
\end{tabu}
\end{center}

计算中间交换电阻的两次测得的$R_x=\sqrt{R_{01}R_{02}}$。其不确定度由两部分合成,一部分是受电桥灵敏度所限的示零误差。这部分的相对不确定度为$\frac{\delta R_x} {R_x}=\frac{0.2}{S}=\num{1.3e-4}$。另一部分是电阻箱读数的允差。\SI{366.1}{\ohm}和\SI{366.2}{\ohm}两个读数均伴有约\SI{.39}{\ohm}的允差,折算成相对不确定度为\num{6.2e-4}。从而$R_x$的相对不确定度为二者的方和根,等于\num{6.3e-4}。故$R_x=\SI{366.1(2)}{\ohm}$。
\section{了解影响直流电桥灵敏度的因素}
改变实验条件,归纳可能影响电桥的灵敏度的因素,如电源电压、检流计内阻(用限流电位器实现)以及桥臂电阻等。
\begin{center}
\begin{tabu} to \linewidth {X[c,-10] X[c] X[c]|X[c] X[c] X[c]|X[c] X[c]}
\hline
电源电压$E/\si{\volt}$	&$R_1/R_2$	&$R_h/\si{\ohm}$	&$R_0/\si{\ohm}$	&$\Delta R_0/\si{\ohm}$	&$\Delta n$	&$R^x_2/\si{\ohm}$	&灵敏度$S$
\\
\hline
4.030	&500\si{\ohm}/500\si{\ohm}	&0	&366.2	&1	&4.1	&366.2	&\num{1.5e3}
\\
4.030	&500\si{\ohm}/5000\si{\ohm}	&0	&3663.0	&100	&8.0	&3663	&\num{3.2e2}
\\
4.030	&500\si{\ohm}/500\si{\ohm}	&2993	&366.2	&10	&5.2	&366.2	&\num{1.9e2}
\\
2.003	&500\si{\ohm}/500\si{\ohm}	&0	&366.2	&3	&6.2	&366.2	&\num{7.6e2}
\\
\hline
\end{tabu}
\end{center}
从表中可以看出,桥臂上两个电阻较接近时,电桥灵敏度较高。电源电压较高,检流计内阻较小时,电桥灵敏度较高。
\section{思考题}
如果电源电压大幅下降,则会降低电桥的灵敏度,增大测量的误差。但如果只是电源电压稍有波动,则不会对测量造成什么影响。

如果测量的电阻小到了导线电阻不可忽略的地步,那么由于导线的电阻会被计入电桥臂上的电阻中,因此会产生一定的误差。

检流计的灵敏度是直接决定电阻测量的精确度的,如果灵敏度不够高势必会对精确度造成影响。如果检流计的零点没有调准,那实际测量时读的零点就并不是真正电桥平衡的零点。如果零点实在调不准,那就只能采取交换检流计正负接线测两次取平均的做法,来消除电桥零点不准所造成的系统误差。

\end{document} 