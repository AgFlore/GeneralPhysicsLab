% !TEX TS-program = Xelatex
% !TEX encoding = UTF-8 Unicode

\documentclass[UTF8]{ctexart}
\usepackage{amsmath}
\usepackage[bottom]{footmisc}
\usepackage{geometry}
\usepackage{pdfpages}
\usepackage{hyperref}
\usepackage{graphicx}
\usepackage{figsize}
\usepackage[separate-uncertainty = true,per-mode=symbol]{siunitx}
\usepackage{tabu}
\usepackage{wasysym}
\geometry{left=0.6in,right=0.6in,bottom=0.6in,top=0.6in}

\title{实验三十六:光的时间相干性}
\author{朱寅杰 1600017721}
\date{2018年3月23日}

\begin{document}
\maketitle
\setcounter{section}{36}
\subsection{各种光源的相干长度的估测}
在迈克耳孙干涉仪上,调出白光(台灯)的等厚干涉。观察到从中心条纹数,只有一条条纹是白色的,其余的均为彩色条纹。白光波长按照\SI{550}{\nm}估算的话,相干长度就只有约$\Delta L_{max}=\SI{550}{\nm}$的量级,相干时间$\Delta L_{max}/c=\SI{2e-15}{\s}$。

如果给台灯加上橙色的滤光片再做等厚干涉,肉眼能分辨出的条纹总数有54条,相当于一边27级。橙光波长按照\SI{625}{\nm}估算,相干长度能有$\Delta L_{max}=\SI{16.9}{\micro\meter}$,相干时间$\Delta L_{max}/c=\SI{5.64e-14}{\s}$。如果换上黄色的滤光片,肉眼能分辨出的条纹总数有108条,相当于一边有54级。黄光波长按照\SI{578}{\nm}估算,相干长度$\Delta L_{max}=\SI{31.2}{\micro\meter}$,相干时间为$\Delta L_{max}=\SI{1.04e-13}{\s}$

如果给汞灯加上黄色的滤光片再做等倾干涉,移动$M_1$镜,从\SI{20.902}{\mm}到\SI{44.583}{\nm}一直都能看到等倾条纹,于是估算汞黄光的相干长度$\Delta L_{max}=\SI{44.583}{\mm}-\SI{20.902}{\mm}=\SI{23.681}{\mm}$,相干时间为$\Delta L_{max}/c=\SI{7.8991e-11}{\s}$。

\subsection{汞灯黄色双线的波长差的测量}
调出汞灯(加黄色滤光片)的等倾干涉,在等光程附近可以观察出条纹可见度随$M_1$镜位置的周期性变化。记录下连续七个可见度最弱的点:
\begin{center}
\begin{tabu}{X[c]|X[c]X[c]X[c]X[c]X[c]X[c]X[c]}
\hline
\#	&1&2&3&4&5&6&7\\
\hline
位置/mm	&33.974	&33.897	&33.817	&33.738	&33.654	&33.576	&33.498\\
\hline
\end{tabu}
\end{center}
做一个最小二乘即可得到相邻两个可见度极弱的点的距离为$\Delta d=\SI{.07975(33)}{\mm}$,相关系数为\num{.99996}。于是用书上的公式(36.4)得到双线波长差为$\Delta\lambda=\frac{\lambda^2}{2\Delta d}=\SI{2.09(1)}{\nm}$。

计数两个可见度极弱点之间光强峰谷数目,得有$\Delta k=272$个峰出现,于是知双线波长差为$\Delta\lambda=\lambda/\Delta k=\SI{2.12}{\nm}$。

\subsection{杂谈}
上文中所有不确定度基本都不作数,因为并不知道干涉仪的轮子的精度有多少。

干涉仪手轮的空程差太过巨大,对于实验的效率造成了极大的影响。不知道现在市面上有没有卖空程差较小的干涉仪呀。
\end{document} 