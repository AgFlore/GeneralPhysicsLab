\documentclass[UTF8]{ctexart}
\usepackage{amsmath}
\usepackage{geometry}
\usepackage{hyperref}

\geometry{left=0.7in,right=0.7in,bottom=0.7in,top=0.7in}
\usepackage{wasysym}
\title{陈斌广相18春期末题回忆}
\date{2018年6月23日}
\begin{document}
\maketitle
\paragraph{I.}已知一个球坐标下的度规$ds^2=A(r)dr^2+B(r)d\theta^2+\sin^2\theta d\phi$,算几个表面积啊体积啊什么的。

\paragraph{II.}史瓦西度规下,一艘飞船沿径向飞向黑洞中心。给出了飞船在飞行过程中速度与位置的关系$dr/dt=-f(r)$($f(r)$是一个给出的一个帮你凑好的并不复杂的关系)。验证飞船的世界线确实是类时的,并计算飞船从$8GM$处飞到黑洞中心的固有时。判断飞船是自由下落还是开了助推器。

\paragraph{III.}史瓦西度规下,一个光源从无穷远静止下落(\href{https://www.bilibili.com/video/av6951241}{\twonotes 让爱~坠入~这深渊里\twonotes})。光源发出的光沿着径向。
\begin{itemize}
  \item 第一问算$R_0$处的4-速度。
  \item 第二问求$R_0$处发出的光子在$36R_s$处静止观察者看来的红移。
  \item 第三问求$R_0$处发出的光子在$6R_s$出的圆周轨道上的运动观察者看来的红移。
\end{itemize}


\paragraph{IV.}科尔度规下,赤道面上圆轨道。
\begin{itemize}
  \item 第一问:用能量($p_t$)和角动量($p_{\phi}$)表示$p^t$和$p^{\phi}$。
  \item 第二问:求(有质量)粒子的圆轨道周期与半径的关系(给了Hint:用$r$-分量的测地线方程)
\end{itemize}

\paragraph{V.}详细阐述真空中传播引力波的物理性质。结合守恒量,说明产生引力辐射的必要条件。计算一个两体的简谐振动系统的质量四极矩和辐射功率(公式均给出)。


\end{document} 