% !TEX TS-program = Xelatex
% !TEX encoding = UTF-8 Unicode

\documentclass[UTF8]{ctexart}
\usepackage{amsmath}
\usepackage[bottom]{footmisc}
\usepackage{geometry}
\usepackage{graphicx}
\usepackage{figsize}
\usepackage[separate-uncertainty = true,per-mode=symbol]{siunitx}
\usepackage{tabu}
\usepackage{wasysym}
\geometry{left=0.5in,right=0.5in,bottom=0.7in,top=0.7in}

\title{实验三十四:全息照相}
\author{朱寅杰 1600017721}
\date{2018年6月1日}

\begin{document}

\maketitle
\setcounter{section}{34}
物光和参考光夹角约为30°,曝光时间\SI{4}{\second}。

将全息片放在原位观察虚像,用手指看像差,确认像就在原来物的位置。像是正立的,与原物大小完全相同。如果从底片不同位置观察的话,看到的像也随之有立体变化,仿佛就是在从不同角度观察原物体一样。如果转动入射的角度,没有观察到像出现明显的变化。对此现象的解释是,根据菲涅尔原理,空间中的波动情况可以由一个波前面上的信息完全决定。如果入射光是球面波,并且入射的球面波的曲率与记录时相同(即光源到全息片距离不变时),经过全息片衍射发出的光就会和原来物体反射发出的光场一模一样,因此从各个角度看到的像都和原物一模一样。原则上转动光入射的角度,所看到的像可能会有畸变。但可能实验时转动的角度并不大,看到的现象并不明显。

改变底片与光源之间的距离,发现光源较近时像小,光源较远时像大。光源较近时,相当于全息片处光波面的曲率半径较小,所成的像也比较小,书上也提到了这个现象。

底片翻转180°,打上共轭光,发现成一个正立近似等大的实像,位置就在全息片右前方十几厘米处吧。这与书上的描述一致。

拿激光直接照射,能观察到:在光线前进的左手和右手方向各在无穷远处成一个实像(在任意位置都能用屏幕接收到清晰的像,并且像的大小与距离成正比)。左手方向的像正立,右手方向的像倒立,有点像是夫琅禾费衍射中+1级和-1级的衍射像。同时在像的右手边能看到一个正立的近似等大的虚像,但是比较弱。激光器直接打出的光可以看作是准直性很好的平行光,猜想可能就是在记录了条纹信息的全息片上产生了性质类似于夫琅禾费衍射的现象吧,然后+1级的像是正立的,-1级的像是倒立的。那个虚像感觉和之前拿球面光衍射时看到的比较相似,可能是由和之前观察虚像时类似的机制形成的,也可能是这个方向有杂光造成的。











\end{document} 